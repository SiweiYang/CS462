\documentclass[12pt]{article}
\pagestyle{empty}

\usepackage{fullpage,amssymb,amsmath}
%\usepackage{cyrillic}
\def\sub{{\rm sub}}
\def\pref{{\rm pref}}
\def\suff{{\rm suff}}
\def\Pref{{\rm Pref}}

\thispagestyle{empty}

\begin{document}
\begin{center}
\large\bf University of Waterloo\\
CS~462 --- Formal Languages and Parsing\\
Winter 2013\\
Problem Set 1\\
\end{center}

\bigskip

\noindent{\it Distributed Tuesday, January 8 2013.}

\smallskip

\noindent{\it Due Tuesday, January 15 2013, in class.}

\bigskip\bigskip

All answers should be accompanied by proofs.  In all problems the underlying
alphabet $\Sigma$ is assumed to be finite.

\begin{enumerate}

\item{} [10 marks]  
Show that for every infinite string $\bf w$ there must be some letter
$a$ and some finite string $x$ such that $axa$ appears infinitely often as a
subword of $\bf w$.

\medskip

\item{} [10 marks]  
Let $u, v$ be strings of the same length, and let
$d(u,v)$ be the number of positions on which $u$ and $v$ differ; that is,
the number of indices $i$ for which $u[i] \not= v[i]$.    
For example, $d({\tt seven},{\tt three}) = 4$.  Show that for all
strings $x,y$ we have $d(xy, yx) \not= 1$.  Note that $x$ and $y$ need
not be of the same length here.

{\it Hint:}  There is a very simple solution that can be hard to find,
and a lengthy solution that is easy to find.

\medskip

\item{} [10 marks]  An infinite string ${\bf x} = a_0 a_1 a_2 \cdots$
is said to be {\sl recurrent} if every
finite subword that occurs in $\bf x$ occurs infinitely often in $\bf x$.
An infinite string is {\sl mirror invariant} if for every finite subword $w$ of
$\bf x$, the reversed string $w^R$ is also a subword of $\bf x$.

\begin{itemize}

\item[(a)]  Show that an infinite string $\bf x$ is recurrent if and only
if every
finite subword that occurs in $\bf x$ occurs at least twice.

\item[(b)]  Show that if an infinite string is mirror invariant,
then it is recurrent.

\end{itemize}


\end{enumerate}

\end{document}

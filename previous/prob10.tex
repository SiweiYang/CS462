\documentclass[12pt]{article}
\pagestyle{empty}

\usepackage{fullpage,amssymb,amsmath}
\usepackage{float}
%\usepackage{cyrillic}
\usepackage{epsfig}
\def\sub{{\rm sub}}
\def\pref{{\rm pref}}
\def\suff{{\rm suff}}
\def\shuff{{\rm shuff}}
\def\Enn{{\mathbb N}}

\thispagestyle{empty}

\begin{document}
\begin{center}
\large\bf University of Waterloo\\
CS~462 --- Formal Languages and Parsing\\
Winter 2011\\
Problem Set 10\\
\end{center}

\bigskip

\noindent{\it Distributed Wednesday, March 16 2011.}

\smallskip

\noindent{\it Due Wednesday, March 23 2011, in class.}

\bigskip\bigskip

All answers should be accompanied by proofs.

\begin{enumerate}

\item{} [10 marks]  
Let $S$ be a subset of $\Enn^2$.
Show that $\{a^i b^j \ : \ (i,j) \in S \}$ is context-free if and only
if $S$ is semilinear.

Is the analogous result true for $\{ a^i b^j c^k \ : \ (i,j,k) \in S \}$
and $S \subseteq \Enn^3$?

\bigskip

\item{} [10 marks] 
Consider the following grammar
\begin{eqnarray*}
S & \rightarrow & A C \ | \ B D \ | \ b \\
A & \rightarrow &  B C \ | \ A S \ | \ a \\
B & \rightarrow &  D A \ | \  c \\
C & \rightarrow &  B B \ | \ a \\
D & \rightarrow &  C A \ | \ a
\end{eqnarray*}

Determine if $aaabcc \in L(G)$ using the CYK algorithm, and
if it is, give a parse tree.  Show your work.
%06

\bigskip

\item{} [10 marks]  
Suppose $L$ is a CFL that is inherently ambiguous.  Then by the definition,
for every
context-free grammar $G$ with $L = L(G)$, at least one word in $L$ has at
least two different parse trees in $G$.
Show that in fact {\it infinitely}
many words in $L$ must have at least two different parse trees.

\end{enumerate}

\end{document}

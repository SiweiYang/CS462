\documentclass[12pt]{article}
\pagestyle{empty}

\usepackage{fullpage,amssymb,amsmath}
%\usepackage{cyrillic}
\def\sub{{\rm sub}}
\def\pref{{\rm pref}}
\def\suff{{\rm suff}}
\def\shuff{{\rm shuff}}

\thispagestyle{empty}

\begin{document}
\begin{center}
\large\bf University of Waterloo\\
CS~462 --- Formal Languages and Parsing\\
Winter 2011\\
Problem Set 4\\
\end{center}

\bigskip

\noindent{\it Distributed Wednesday, January 26 2011.}

\smallskip

\noindent{\it Due Wednesday, February 2 2011, in class.}

\bigskip\bigskip

All answers should be accompanied by proofs.

\begin{enumerate}

\item{} [10 marks]
Which of the following assertions
are true for all languages $L_1, L_2$?
\begin{itemize}
\item[(a)] $(L_1/L_2)L_2 \subseteq L_1$;

\item[(b)] $L_1\subseteq (L_1/L_2)L_2$.
\end{itemize}

\item{} [10 marks]
Let $L_1, L_2 \subseteq \Sigma^*$ be languages.  In class we defined the
(right) quotient $L_1/L_2$.  But there is also a (left)
quotient defined by
$$L_1 \backslash L_2 = \lbrace x \in \Sigma^* \ : \ 
	\text{ there exists } y \in L_2 \text{ such that }
		yx \in L_1 \rbrace .$$
Prove or disprove: if $L_1$ is regular, and $L_2$ is any language,
then $L_1 \backslash L_2$ is regular.

\item{} [10 marks]
Consider the shuffle operation
{\rm shuff} as defined in Example 3.3.8, page 57.
If ${\rm shuff}(L, \lbrace 0 \rbrace)$ is regular, need $L$ be regular?

\end{enumerate}

\end{document}

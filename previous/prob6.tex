\documentclass[12pt]{article}
\pagestyle{empty}

\usepackage{fullpage,amssymb,amsmath}
%\usepackage{cyrillic}
\def\sub{{\rm sub}}
\def\pref{{\rm pref}}
\def\suff{{\rm suff}}
\def\shuff{{\rm shuff}}

\thispagestyle{empty}

\begin{document}
\begin{center}
\large\bf University of Waterloo\\
CS~462 --- Formal Languages and Parsing\\
Winter 2011\\
Problem Set 6\\
\end{center}

\bigskip

\noindent{\it Distributed Wednesday, February 9 2011.}

\smallskip

\noindent{\it Due Wednesday, February 16 2011, in class.}

\bigskip\bigskip

All answers should be accompanied by proofs.

\begin{enumerate}

\item{} [10 marks]
Is the Myhill-Nerode equivalence relation induced by a language
$L$ the same as the one induced by $\overline{L}$, the complement of $L$?  
Explain.

\item{} [10 marks] 
Consider the following transformation on languages:
$$ {\tt sqrt} (L) = \lbrace x \in \Sigma^* \ : \ \text{ there exists }
y \in \Sigma^* \text{ such that } |y| = |x|^2 \text{ and } xy \in L
\rbrace.$$

Show that if $L$ is regular, then so is ${\tt sqrt}(L)$.  Hint:  use
the boolean matrix approach.

\item{} [10 marks]
Show that if $L$ is a regular language, then so is
$${\tt ROOT}(L) = \lbrace w \ : w^{|w|} \in L \rbrace.$$
Hint:  use the transformation automaton.

\end{enumerate}

\end{document}

\documentclass[12pt]{article}
\pagestyle{empty}

\usepackage{fullpage,amssymb,amsmath}
\usepackage{float}
%\usepackage{cyrillic}
\usepackage{epsfig}
\def\sub{{\rm sub}}
\def\pref{{\rm pref}}
\def\suff{{\rm suff}}
\def\shuff{{\rm shuff}}

\thispagestyle{empty}

\begin{document}
\begin{center}
\large\bf University of Waterloo\\
CS~462 --- Formal Languages and Parsing\\
Winter 2011\\
Problem Set 8\\
\end{center}

\bigskip

\noindent{\it Distributed Wednesday, March 2 2011.}

\smallskip

\noindent{\it Due Wednesday, March 9 2011, in class.}

\bigskip\bigskip

All answers should be accompanied by proofs.

\begin{enumerate}

\item{} [10 marks]
Consider the language $L$ consisting of all subwords of the Thue-Morse
word.  Show that $L$ is not context-free.  Hint:  the Thue-Morse word
avoids cubes.

\item{} [10 marks] 
Consider, for each $a \in \Sigma$, the new letter $a'$.
The set of all such letters is $\Sigma'$.  For a word $x \in (\Sigma \ \cup \
\Sigma')^*$, define the reduced word $r(x)$ obtained by treating
$a'$ as an inverse to $a$ satisfying $aa' = a'a = \epsilon$, and applying
the transformations $aa' \rightarrow \epsilon$ and
$a'a \rightarrow \epsilon$ repeatedly until no letter is next to its inverse.
For example, $r(c b' a a' b a b b') = c a$, because
$cb'aa'babb' \rightarrow cb'babb' \rightarrow cabb' \rightarrow ca$.

\medskip

Show that if $L \subseteq (\Sigma \ \cup \ \Sigma')^*$ is context-free, then
$r(L)$ need not be context-free.  Hint:  show that
that we can define the quotient of two languages in terms of the operation
$r$ and other operations that preserve context-freeness.

\item{} [10 marks]
Let $k$ be a fixed integer $\geq 2$.  
Given a fixed alphabet $\Sigma$,
define $L_k = \lbrace w^k \ : \ w \in \Sigma^* \rbrace$ to be the set
of all $k$'th powers of words in $\Sigma^*$.  Show that the
complement $\Sigma^* - L_k$ is always context-free.  This is the
language of ``non-$k$th-powers''.

\medskip

Hint:  on input $x$, use counters on the stack to keep track 
of some quantity that depends on both the length of $x$ and the distance
between two symbols in $x$ that are different.  Yes, I know this is a vague
hint.

\end{enumerate}

\end{document}

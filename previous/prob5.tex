\documentclass[12pt]{article}
\pagestyle{empty}

\usepackage{fullpage,amssymb,amsmath}
%\usepackage{cyrillic}
\def\sub{{\rm sub}}
\def\pref{{\rm pref}}
\def\suff{{\rm suff}}
\def\shuff{{\rm shuff}}

\thispagestyle{empty}

\begin{document}
\begin{center}
\large\bf University of Waterloo\\
CS~462 --- Formal Languages and Parsing\\
Winter 2011\\
Problem Set 5\\
\end{center}

\bigskip

\noindent{\it Distributed Wednesday, February 2 2011.}

\smallskip

\noindent{\it Due Wednesday, February 9 2011, in class.}

\bigskip\bigskip

All answers should be accompanied by proofs.

\begin{enumerate}

\item{} [10 marks]
Suppose $L \subseteq \Sigma^*$ is a regular language.  Show that
$$2L := \lbrace a_1 a_1 a_2 a_2 \cdots a_k a_k \ : \ 
\text{each $a_i \in \Sigma$ and $a_1 a_2 \cdots a_k \in L$} \rbrace$$
is regular.

\item{} [10 marks] Let $h: \Sigma^* \rightarrow \Sigma^*$ be a morphism,
and let $L \subset \Sigma^*$ be a language.  Define
$$h^{-*} (L) = \bigcup_{i \geq 0} h^{-i} (L),$$
where by $h^{-i}(L)$ we mean $\overbrace{h^{-1} ( h^{-1} ( \cdots h^{-1}}^{i
 \text{ times}} (L)))$.
Show that if $L$ is regular, so is $h^{-*} (L)$.

\item{} [10 marks]
\begin{itemize}
\item[(a)]  [2 marks -- the easy part]
Let $h: \lbrace a,b \rbrace^* \rightarrow \lbrace 0 \rbrace^*$
be the morphism defined by $h(a) = h(b) = 0$.
Let $L \subseteq \lbrace a,b \rbrace^*$ be a language.
If $h(L)$ is regular, need
$L$ be regular?  

\item[(b)]  [8 marks -- the harder part] Now let $\Sigma = \lbrace a,b, c \rbrace$.
Let $h_{a,b}: \Sigma^* \rightarrow (\Sigma \cup \lbrace 0 \rbrace)^*$ be the morphism
defined by $h(a) = h(b) = 0$ and $h(c) = c$.  
Thus $h_{a,b}$ maps both $a$ and $b$ to $0$ and leaves the remaining letter,
$c$, unchanged.  In a similar way define
$h_{a,c}$ and $h_{b,c}$.  
\medskip

Now let $L \subseteq \Sigma^* $ be a language, and define
$A := h_{b,c}(L)$, $B := h_{a,c}(L)$, and $C := h_{a, b}(L)$.
If $A$, $B$, and $C$ are all regular, need $L$ be regular?

\end{itemize}

\end{enumerate}

\end{document}

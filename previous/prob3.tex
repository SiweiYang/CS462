\documentclass[12pt]{article}
\pagestyle{empty}

\usepackage{fullpage,amssymb,amsmath}
\def\sub{{\rm sub}}
\def\pref{{\rm pref}}
\def\suff{{\rm suff}}
\def\shuff{{\rm shuff}}

\thispagestyle{empty}

\begin{document}
\begin{center}
\large\bf University of Waterloo\\
CS~462 --- Formal Languages and Parsing\\
Winter 2011\\
Problem Set 3\\
\end{center}

\bigskip

\noindent{\it Distributed Wednesday, January 19 2011.}

\smallskip

\noindent{\it Due Wednesday, January 26 2011, in class.}

\bigskip\bigskip

All answers should be accompanied by proofs.

\begin{enumerate}

\item{} [10 marks]
What is the largest power of a word
that you can find in the decimal expansion of $\pi$?  Use a search
engine to locate digits of $\pi$ and any method to search for powers,
give the power, its location in $\pi$, and the URL you used.
Note:  I was able to find a $9$th power without any programming at
all, and just 5 minutes with a search engine.  Can you do better?


\item{} [10 marks]
A word $w$ is a {\it conjugate} of a word $x$ if $w$ can be obtained
from $x$ by cyclically shifting the letters.  For example,
{\tt enlist} is a conjugate of {\tt listen}.

(a) [7 marks] Let $y, z$ be palindromes.  Show that if at least
one of $|y|, |z|$ is even, then some conjugate of $yz$ is a palindrome.

(b) [3 marks] Give an example to show that
if both $|y|$ and $|z|$ are odd, then it is possible to have no
conjugate of $yz$ be a palindrome.

\item{} [10 marks]   Prove the following improvement on Theorem~2.3.6
in the course text.  You can use the same idea as in the proof of
that theorem.

Let $x$ and $y$ be nonempty words.

Show that 
$x^\alpha = y^\beta$ 
for some fractional exponents $\alpha, \beta$ satisfying
$\alpha + \beta \leq \alpha \beta$ iff 
$xy = yx$.


\end{enumerate}

\end{document}
